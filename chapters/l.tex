\documentclass{standalone}
\begin{document}
\newpage
\section{Lösungsvorschläge}
\subsection{Aufgabe 1}
\begin{enumerate}[a)]
\item
    \begin{minted}{c}
int main(void) {
    double initial_v_x = 0.0;
    printf("Geben sie die horizontale Startgeschwindigkeit an\n");
    scanf(" %lf", &initial_v_x);

    double initial_v_y = 0.0;
    printf("Geben sie die vertikale Startgeschwindigkeit an\n");
    scanf(" %lf", &initial_v_y);

    double t = 0.0;
    printf("Geben sie eine Zeit an\n");
    scanf(" %lf", &t);

    double pos_x = initial_v_x * t;
    double pos_y = initial_v_y * t + (-9.81 / 2.0) * t * t;

    printf("Position des Objekts: (%.2lf, %.2lf)\n", pos_x, pos_y);
}
    \end{minted}

\item
    \begin{minted}{c}
struct Vec2 {
    double x;
    double y;
};

int main(void) {
    struct Vec2 initial_v = { 0.0, 0.0 };
    printf("Geben sie die horizontale Startgeschwindigkeit an\n");
    scanf(" %lf", &initial_v.x);
    printf("Geben sie die vertikale Startgeschwindigkeit an\n");
    scanf(" %lf", &initial_v.y);

    double t = 0.0;
    printf("Geben sie eine Zeit an\n");
    scanf(" %lf", &t);

    struct Vec2 pos = {
        initial_v.x * t,
        initial_v.y * t + (-9.81 / 2.0) * t * t
    };

    printf("Position des Objekts: (%.2lf, %.2lf)\n", pos.x, pos.y);
}
    \end{minted}

\item
    \begin{minted}{c}
struct Vec2 {
    double x;
    double y;
};

int main(void) {
    struct Vec2 initial_v = { 0.0, 0.0 };
    printf("Geben sie die horizontale Startgeschwindigkeit an\n");
    scanf(" %lf", &initial_v.x);
    printf("Geben sie die vertikale Startgeschwindigkeit an\n");
    scanf(" %lf", &initial_v.y);

    // initial_v.y + (-9.81) * t = 0
    double t = initial_v.y / 9.81;

    struct Vec2 pos = {
        initial_v.x * t,
        initial_v.y * t + (-9.81 / 2.0) * t * t
    };

    printf("Höchster Punkt: (%.2lf, %.2lf)\n", pos.x, pos.y);
}
    \end{minted}

\item
    \begin{minted}{c}
struct Vec2 {
    double x;
    double y;
};

int main(void) {
    struct Vec2 initial_v = { 0.0, 0.0 };
    printf("Geben sie die horizontale Startgeschwindigkeit an\n");
    scanf(" %lf", &initial_v.x);
    printf("Geben sie die vertikale Startgeschwindigkeit an\n");
    scanf(" %lf", &initial_v.y);

    // initial_v.y * t + (-9.81 / 2.0) * t * t = 0
    double t = 2.0 * initial_v.y / 9.81;

    struct Vec2 pos = {
        initial_v.x * t,
        initial_v.y * t + (-9.81 / 2.0) * t * t
    };

    if (9.5 < pos.x && pos.x < 10.5) {
        printf("Treffer!\n");
    } else {
        printf("Verfehlt.\n");
    }
}
    \end{minted}

\end{enumerate}
\subsection{Aufgabe 2}
\begin{enumerate}[a)]
\item
    \begin{minted}{c}
int main(void) {
    char text[101] = { 0 };
    printf("Verschlüsselten Text eingeben\n");
    scanf(" %100s", text); // besser wäre "fgets(text, 100, stdin)", da scanf keine Leerzeichen erlaubt

    int key = 0;
    printf("Schlüssel eingeben\n");
    scanf(" %i", &key);

    for (int i = 0; text[i] != 0; i++) {
        char enc = text[i];
        if (0x41 <= enc && enc <= 0x5A) { // Großbuchstaben
            if (enc - key < 0x41) { // Underflow
                text[i] = enc - key + 26;
            } else if (enc - key > 0x5A) { // Overflow
                text[i] = enc - key - 26;
            }  else {
                text[i] = enc - key;
            }
        } else if (0x61 <= enc && enc <= 0x7A) { // Kleinbuchstaben
            if (enc - key < 0x61) { // Underflow
                text[i] = enc - key + 26;
            } else if (enc - key > 0x7A) { // Overflow
                text[i] = enc - key - 26;
            } else {
                text[i] = enc - key;
            }
        }
    }

    printf("Entschlüsselter Text: %s\n", text);
}
    \end{minted}

\item
    \begin{minted}{c}
int main(void) {
    int mode = 0;
    while (!(mode == 1 || mode == 2)) {
        printf("Modus auswählen\n");
        printf("\t1: Entschlüsseln\n");
        printf("\t2: Verschlüsseln\n");
        scanf(" %i", &mode);
    }

    char text[101] = { 0 };
    if (mode == 1) {
        printf("Verschlüsselten Text eingeben\n");
    } else {
        printf("Zu verschlüsselnden Text eingeben\n");
    }
    scanf(" %100s", text);

    int key = 0;
    printf("Schlüssel eingeben\n");
    scanf(" %i", &key);

    if (mode == 2) {
        key = -key;
    }

    for (int i = 0; text[i] != 0; i++) {
        char enc = text[i];
        if (0x41 <= enc && enc <= 0x5A) {
            if (enc - key < 0x41) {
                text[i] = enc - key + 26;
            } else if (enc - key > 0x5A) {
                text[i] = enc - key - 26;
            }  else {
                text[i] = enc - key;
            }
        } else if (0x61 <= enc && enc <= 0x7A) {
            if (enc - key < 0x61) {
                text[i] = enc - key + 26;
            } else if (enc - key > 0x7A) {
                text[i] = enc - key - 26;
            } else {
                text[i] = enc - key;
            }
        }
    }

    if (mode == 1) {
        printf("Entschlüsselter Text: %s\n", text);
    } else {
        printf("Verschlüsselter Text: %s\n", text);
    }
}
    \end{minted}

\end{enumerate}
\subsection{Aufgabe 3}
\begin{enumerate}[a)]
\item
    \begin{minted}{c}
int main(void) {
    const double w = 0.0000004;
    const double l = 0.00000025;
    const double u_gs = 2.0;
    const double i_d = 0.00015;
    const double u_t = 0.4;
    const double k = 0.000128;

    const double epsilon = 1e-5;

    if (u_gs < u_t) {
        printf("Der Transistor ist im Sperrbereich\n");
    } else {
        const double beta = k * w / l;
        const double i_d_sat = (beta / 2.0) * (u_gs - u_t) * (u_gs - u_t); // Sah-Modellgleichung

        if (i_d - epsilon <= i_d_sat && i_d_sat <= i_d + epsilon) { // Nicht "i_d == i_d_sat"!
            printf("Der Transistor ist im Sättigungsbereich\n");
        } else {
            printf("Der Transistor ist im linearen Bereich\n");
        }
    }
}
    \end{minted}

\item
    \begin{minted}{c}
int main(void) {
    const double w = 0.0000004;
    const double l = 0.00000025;
    const double u_gs = 2.0;
    const double i_d = 0.00015;
    const double u_t = 0.4;
    const double k = 0.000128;

    const double epsilon = 1e-5;

    if (u_gs < u_t) {
        printf("Der Transistor ist im Sperrbereich\n");
        printf("u_ds >= 0\n");
    } else {
        const double beta = k * w / l;
        const double i_d_sat = (beta / 2.0) * (u_gs - u_t) * (u_gs - u_t);

        if (i_d - epsilon <= i_d_sat && i_d_sat <= i_d + epsilon) {
            printf("Der Transistor ist im Sättigungsbereich\n");
            printf("u_ds >= %lf\n", u_gs - u_t);
        } else {
            const double u_ds_lin = (-(2.0 * (u_t - u_gs)) - sqrt((2.0 * (u_t - u_gs)) * (2.0 * (u_t - u_gs)) - (4.0 * (2.0 * i_d / beta)))) / 2.0; // Sah-Modellgleichung

            printf("Der Transistor ist im linearen Bereich\n");
            printf("u_ds = %lf\n", u_ds_lin);
        }
    }
}
    \end{minted}

\item
    \begin{minted}{c}
int main(void) {
    const double u_gs = 2.0;
    const double u_ds = 0.55;
    const double w = 0.0000004;
    const double l = 0.00000025;
    const double u_t = 0.4;
    const double k = 0.000128;

    const double beta = k * w / l;

    if (u_gs < u_t && 0.0 < u_ds) {
        printf("Der Transistor ist im Sperrbereich\n");
        printf("i_d = 0\n");
    } else if (u_gs > u_t && 0.0 < u_ds && u_ds < u_gs - u_t) {
        const double i_d = beta * (u_gs - u_t - (u_ds / 2.0)) * u_ds;

        printf("Der Transistor ist im linearen Bereich\n");
        printf("i_d = %lf\n", i_d);
    } else if (u_gs > u_t && u_ds > u_gs - u_t) {
        const double i_d = (beta / 2.0) * (u_gs - u_t) * (u_gs - u_t);

        printf("Der Transistor ist im Sättigungsbereich\n");
        printf("i_d = %lf\n", i_d);
    } else {
        printf("Der Transistor bricht die Gesetze der Physik\n");
        printf("i_d = ???\n");
    }
}
    \end{minted}

\end{enumerate}
\end{document}
