\documentclass{standalone}
\begin{document}
\section{Aufgabe 1}
    Nachdem du Herr Meidner zum dritten mal gefragt hast, ob ein Diagramm als Beweis gilt, möchte er dich mit dem Stift seines Tablets bewerfen. Hilf ihm dabei indem du ein Programm schreibst, das den Wurf eines Objekts in einem zwei-dimensionalen Kartischem Koordinatensystem simuliert. Nimm an, dass die Beschleunigung durch die Schwerkraft $g=9.81\text{ms}^-2$, die Startposition des Balls $\begin{bmatrix}x_0\\y_0\end{bmatrix}=\begin{bmatrix}0\text{m}\\0\text{m}\end{bmatrix}$ und ignoriere jegliche Reibungskräfte.

\begin{enumerate}[a)]
\item
    Berechne mit einem Programm die Position des Objekts zu einer bestimmten Zeit. Frag  dafür den Nutzer nach einem Geschwindigkeitsvektor $\begin{bmatrix}v_{x0}\\v_{y0}\end{bmatrix}$ für das Objekt und einer Zeit. Les diese Werte mit \mintinline{C}{scanf} als \mintinline{C}{double} ein. Verwende dann zur Berechnung die Gesetze der Kinematik. Gib anschließend die Werte mit zwei Nachkommastellen aus (so genau kann Herr Meidner nun auch nicht zielen).
    \begin{align*}
        x(t)=x_0+v_0t+\frac{at^2}{2}
    \end{align*}

\item
    Schreib dein Programm um, indem du ein \mintinline{C}{struct} nutzt, um zwei-dimensionale Vektoren sinnvoll darzustellen.

\item
    Berechne den höchsten Punkt, den der Ball für den vom Nutzer eingegebenen Geschwindigkeitsvektor erreicht. Tipp: Am höchsten Punkt der Parabel ist die vertikale Geschwindigkeit des Balls gleich Null.

\item
    Nun kannst du das eigentliche Stift-Wurf-Spiel schreiben. Das Ziel des Spiels ist es einen Geschwindigkeitsvektor einzugeben, der dafür sorgt, dass das Objekt in einem bestimmten Bereich landet. Dazu solltest du den Punkt an dem der Ball auf der x-Achse aufkommt berechnen. Teil dann dem Spieler mit, ob das Ziel getroffen wurde oder nicht. Den Zielbereich kannst du in deinem Programm beliebig wählen (beispielsweise das Intervall $]9.5,10.5[$). Und? Hat er getroffen? Aua!

\end{enumerate}
\end{document}
