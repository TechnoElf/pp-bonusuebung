\documentclass{standalone}
\begin{document}
\section{Aufgabe 2}
"ysx ruwkuiiu qbbu husxj xuhpbysx pk isxqbjkdwijxuehyu"\\ \\
Du schaust dir mit doppelter Geschwindigkeit ein Vorlesungsvideo von Herrn Joham an. Du verstehst allerdings kein 
einziges Wort und seine Zeichnungen und Gleichungen sehen so aus, als könne man Geister damit heraufbeschwören. 
Es fühlt sich an wie ein Fiebertraum. Auch das Herabsetzen der Geschwindigkeit auf das 
normale Tempo ändert nichts daran. Plötzlich hörst du die sanfte Stimme von Herrn Zwick in der Ferne. Er schlägt dir vor, 
ein Programm zur Entschlüsselung des Buchstabensalats zu erstellen. "Potzblitz!" denkst du dir, während dein PC automatisch hochfährt
und der Editor gestartet wird.
\subsection{}
Erstelle ein Programm, welches den Benuter als erstes zur Eingabe eines verschlüsselten Textes auffordert. 
Anschließend soll das Programm nach einer Zahl fragen, um die die einzelnen Buchstaben des verschlüsselten Textes
verschoben werden sollen. Der daraus resultierende Text soll ausgegeben werden.
\subsection{}
Erweitere das das Programm nun so, dass der Benutzer zuerst gefragt wird, ob ein Text ent- oder verschlüsselt werden soll.
Dementsprechend soll nun die Möglichkeit hinzugefügt werden, einen vom Benutzer eingegebenen Text zu verschlüsseln, indem 
die Buchstaben um eine bestimmte Anzahl an Buchstaben im Alphabet verschoben werden.\\ \\
Beispiel: Aus 'hallo' wird 'ibmmp', wenn man die Buchstaben um 1 verschiebt.
\end{document}
