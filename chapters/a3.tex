\documentclass{standalone}
\begin{document}
\section{Aufgabe 3}
    In der Zentralübungsaufgabe 2.1.1 in Digitaltechnik sind folgende Werte für einen 
    n-MOS-Transistor gegeben: \\
    Transistorabmessungen: $W = 0,4\mu m$, $L = 0,25\mu m$.\\
    Arbeitspunkt: $U_{GS} = 2V, I_D = 150\mu A$.\\
    Thresholdspannung: $U_t = 0,4V$.\\
    Transistorkostante: $K' = 128 \frac{\mu A}{V^2}$.

\begin{enumerate}[a)]
\item
    Das Tablet von Herrn Maurer hat nach der letzten Zentralübung unerwartet den Geist aufgegeben. Niemand
    kann sich vorstellen, wie es dazu kommen konnte...  Um das Gerät wieder zum laufen zu bekommen, braucht er deine Hilfe. 
    Er befürchtet nämlich, dass die Transistoren vielleicht Schaden genommen haben könnten und nun womöglich im falschen Arbeitsbrereich laufen.
    Stichprobenartig hat er ein paar Messungen vorgenommen (siehe oben). Schreibe ein Programm, welches nach Angabe der dafür benötigten 
    Werte bestimmt in welchem Arbeitsbereich sich der n-MOS Transistor befindet. 
    
\item 
    Dank dir läuft das Tablet nun wieder, allerdings hat Herr Maurer gefallen daran gefunden, die Aufgaben vom Computer lösen zu lassen.
    Erweitere das Programm nun so, dass auch die Aufgabe 2.1.2 gelöst werden kann. In dieser Aufgabe soll mit Hilfe der oben stehenden Werte
    die Drain-Source-Spannung $U_{DS}$ berechnet werden. (Ergebnis: $U_{DS} = 0,55V$)

\item
    Nach kurzer Euphorie, dass nun auch dieses Aufgabe in Windeseisle vom Programm gelöst werden kann, setzt der Größenwahn ein und es fliegen
    plötzen alle möglichen Größen durch den Raum:
    "$V_{Sonne}$, $E_{kin}$, $i_D$, $c_0$, ..."
    "$i_D$?! Das haben Sie doch schon gemessen, oder? Naja, was solls, was ergibt denn hier schon noch Sinn."
    Da du nicht den Zorn des Maurers auf dich ziehen willst, aber langsam hungrig wirst, entscheidest du dich, das Programm 
    wenigstens noch $i_D$ ausrechenen zu lassen.  
\end{enumerate}

Nachdem dein Programm nun den Arbeitsbereich, $U_{DS}$ und $i_D$ ausrechnen kann, erhälst du eine herzerwärmende Mail von Herrn Maurer, in der er dir 
für deine großartige Arbeit dankt. Kurz darauf klingelt Elon Musk an der Tür und bietet dir einem Job an.

Du machst die Augen auf und musst ernüchtert feststellen, dass es nur der Wecker war, der geklingelt hat und dein Vater, der in der Tür steht und fragt wie du 
jemals einen Job bekommen sollst, wenn du nicht bald deinen Schlafrythmus in den Griff bekommst und konsequent für die Prüfungen lernst.
Schade eigentlich.

\end{document}
