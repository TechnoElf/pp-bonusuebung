\documentclass{standalone}
\begin{document}

\section{Aufgabe 3}
In der Zentralübungsaufgabe 2.1.1 in Digitaltechnik sind folgende Werte für einen 
n-MOS-Transistor gegeben: \\
Transistorabmessungen: $W = 0,4\mu m$, $L = 0,25\mu m$.\\
Arbeitspunkt: $U_{GS} = 2V, I_D = 150\mu A$.\\
Thresholdspannung: $U_t = 0,4V$.\\
Transistorkostante: $K' = 128 \frac{\mu A}{V^2}$.

\subsection{}
Schreibe ein Programm, welches nach Angabe der dafür benötigten Werte bestimmt 
in welchem Arbeitsbereich sich der n-MOS Transistor befindet. 
Zur Überprüfung des Programms können die obigen Werte verwendet werden.

\end{document}